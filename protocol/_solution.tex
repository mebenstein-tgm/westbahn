\section{Umsetzung}

\subsection{Entities}

Aus dem UML Diagramm wurden Java Klassen generiert. Zu diesen wurden die Notwendigen Annotationen hinzugefügt. 

\subsubsection{Annotationen}

	Hibernate bietet zur Konfiguration der Entities Annotationen an. 
	
	\begin{itemize}
		\item[] @Entity: wird verwendet um Klassen als Entitäten zu markieren
		\item[] @Id: legt Attribute als Primary Key fest
		\item[] @GeneratedValue: erstellt automatisch ID Werte
		\item[] @Column: kann besondere eigenschaften für die Tabellenspalte festlegen
		\subitem name: legt den Namen fest
		\subitem unique: stellt sicher, dass dieser Wert eindeutig ist
		\item[] @Transient: speichert Attribute nicht in der Datenbank
		\item[] @OneToOne: eindeutige Beziehung zwischen Enitäten
		\item[] @OneToMany: legt fest das eine Collection von einer Enität zu mehreren anderen ist
		\item[] @ManyToOne: legt fest das mehrere Entitäten eine Beziehun zu einer Enität haben
		\item[] @ManyToMany: erstellt Beziehungen zwischen Collections von Entitäten und mehreren Entitäten
		\subitem cascade: Legt fest wie die Beziehungen behandelt werden sollen
		  
	\end{itemize}

	\begin{listing}
	\begin{code}[]{java}
	public class Benutzer {
		
		@Id
		@GeneratedValue(strategy = GenerationType.AUTO)
		private Long ID;
		
		@NotNull
		private String vorName;
		
		@NotNull
		@CorrectEmailConstraint
		private String eMail;
		
		@NotNull
		private String passwort;
		
		@OneToMany(cascade = CascadeType.ALL)
		private Set<Ticket> tickets;
		
		@OneToMany(cascade = CascadeType.ALL)
		private Set<Reservierung> reservierungen;
	}
	\end{code}
	\caption{Entity Klasse}
	\end{listing}

	\clearpage

	\subsection{XML-Mapping}
	
	Bis auf die Klasse ''Reservierung'' wurden alle mit Annotationen erstellt. ''Reservierung'' wurde mit XML-Mapping erstellt. Dies bietet die Möglichkeit unabhängig von der Klassendatei das Mapping festzulegen. Weiters ist es sinnvoll, wenn es schon eine bestehende Datenbankstruktur gibt.
	
	Das Mapping für die Klasse ''xyz'' muss in der Datei ''xyz.hbm.xml'' festgelegt werden.
	
	\begin{itemize}
		\item[] <hibernate-mapping>: Erstellt ein neues Hibernate mapping
		\item[] <class>: Legt fest für welche Klasse das Mapping gilt
		\subitem name: Name bzw. Ort der Klasse. Z.b. entity.Reservierung
		\subitem table: Legt den Name der Datenbanktabelle fest
		\subitem <id: Legt Primärschlüssel fest
		\subsubitem name: Name des Attributs
		\subsubitem <generator>: Legt die Art der Generierung fest
		\subitem <property>: Legt das Mapping für ein Attribut fest
		\subsubitem name: Name des Attributs
		\subsubitem column: Spaltenname
		\subsubitem type: Klassenname oder Ort
		\subitem <many-to-one>: So wie property, nur für Many to One. Syntax gilt auch für alle anderen Kardinalitäten.
	\end{itemize}
	
	\begin{listing}
	\begin{code}[]{java}
		<?xml version = "1.0" encoding = "utf-8"?>
		<!DOCTYPE hibernate-mapping PUBLIC
		"-//Hibernate/Hibernate Mapping DTD//EN"
		"http://www.hibernate.org/dtd/hibernate-mapping-3.0.dtd">
		
		<hibernate-mapping>
		<class name = "entity.Reservierung" table = "Reservierung">
		
		<meta attribute = "class-description">
		Sis is a reserwayschion.
		</meta>
		
		<id name = "ID" type = "java.lang.Long" column = "id">
		<generator class="identity"/>
		</id>
		
		<property name = "datum" column = "datum" type = "java.util.Date"/>
		<property name = "praemienMeilenBonus" column = "praemienMeilenBonus" type = "int"/>
		<property name = "preis" column = "preis" type = "int"/>
		<property name = "status" column = "status" type = "entity.StatusInfo"/>
		<many-to-one cascade="all" name = "benutzer" class = "entity.Benutzer"/>
		<many-to-one cascade="all" name = "strecke"  class = "entity.Strecke"/>
		<many-to-one cascade="all" name = "zug" class = "entity.Zug"/>
		
		</class>
		</hibernate-mapping>
	\end{code}
	\caption{XML-Mapping Reservierung}
	\end{listing}
	
	\section{Setup}
	
	Das Projekt wurde mit Maven umgesetz.
	
	\subsection{Maven}
	
	Für Maven wurden folgende Abhängigkeiten in  ''pom.xml'' definitert.
	
	\begin{itemize}
		\item org.hibernate:hibernate-core:5.3.1.Final
		\item com.h2database:h2:1.4.197
		\item org.hibernate.validator:hibernate-validator:6.0.10.Final
		\item org.apache.maven.plugins:maven-surefire-plugin:2.21.0
		\item junit:junit:4.12
		\item org.glassfish:javax.el:3.0.1-b08
	\end{itemize}

	\subsection{Hibernate}
	
	Für Hibernate wurde ein ''hibernate.cfg.xml'' erstellt. 
	
	\subsubsection{H2}
	
	Damit Hibernate mit der H2 Datenbank kommunizieren kann wurde folgendes Konfiguriert:
	
	\begin{listing}
	\begin{code}[]{java}
		<hibernate-configuration>
		<session-factory>
		<property name="hibernate.connection.driver_class">org.h2.Driver</property>
		<property name="connection.url">jdbc:h2:mem:test</property>
		<property name="connection.username">sa</property>
		<property name="connection.password"/>
		<property name="hibernate.dialect">org.hibernate.dialect.H2Dialect</property>
	\end{code}
	\caption{Hibernate config}
	\end{listing}
	
	\subsubsection{Entitäten}
	
	Damit die Entitäten von Hibernate erkannt werden wurde folgendes Konfiguriert:
	
	\begin{listing}
	\begin{code}[]{java}
		<mapping class="entity.Bahnhof"/>
		<mapping class="entity.Benutzer"/>
		<mapping class="entity.Einzelticket"/>
		<mapping class="entity.Reservierung"/>
		<mapping class="entity.Strecke"/>
		<mapping class="entity.Zeitkarte"/>
		<mapping class="entity.Zug"/>
		<mapping class="entity.Sonderangebot"/>
		
		<mapping resource="Reservierung.hbm.xml" />
	\end{code}
	\caption{Entity Mapping config}
	\end{listing}

	\section{Hibernate}
	\subsection{Session}
	Um ein Sessionobjekt zu erstellen, wurd eine Sessionfactory erstellt.
	
	\begin{listing}
	\begin{code}[]{java}
		return new Configuration().configure("hibernate.cfg.xml").buildSessionFactory();
	\end{code}
	\caption{Erstellen Session Factory}
	\end{listing}

	\subsection{Persistierung}
	Damit ein Objekt persistiert werden kann muss eine Transaktion erstellt werden. In dieser kann dann die Funktion ''persist'' aufgerufen werden, um eine Entität zu Speichern. 
	Damit nicht alle Attribute, welche Entitäten sind, einzeln Persistiert werden müssen wurde zu allen ''cascade = CascadeType.ALL'' hinzugefügt.
	
	\begin{listing}
	\begin{code}[]{java}
		session = HibernateUtil.getSessionFactory().openSession();
		
		for(int i = 0; i < 20;++i) {
			session.beginTransaction();
			Benutzer b = FakeDataGenerator.generateRandomBenutzer();
			b = FakeDataGenerator.addReservationsToBenutzer(b);
			benutzers.add(b);
			
			session.persist(b);
			session.getTransaction().commit();
		}
	\end{code}

	\caption{Persistierung}
	\end{listing}


	\subsection{Validierung}
	
	Mittels Validierungsannotationen können Attribute oder Klassen auf korrektheit überprüft werden.
	
	\begin{itemize}
		\item @NotNull: Attribut darf nicht ''null'' sein
		\item @Email: Validiert eine Email pattern
		\item @Pattern: Regex überprüfung von Strings
		\item @Past: Datum muss in der Vergangenheit liegen
		\item @Future: Datum miss in der Zukunft liegen
		\item @FutureOrPresent/@PastOrPresent
	\end{itemize}
	
	\clearpage
	
	\subsection{Named Queries}
	
	\subsubsection{Syntax von HQL}
	
	\begin{itemize} 
		\item SELECT: ''SELECT''
		\item Spalten: ''spaltenname'' 
		\item AS: ''Benutzer ben'' legt eine Kurzbezeichnung für die Klasse fest
		\item Subqueries: Sind in ''IN'' möglich
		\item Where: Gleich wie in SQL
		\item Parameter: '':varname'' kennzeichnet einen Parameter
	\end{itemize}
	
	\subsection{Definition}
	Die Named-Queries können entweder in ''hibernamte.cfg.xml'' festgelegt werden, order mit Annotationen.
	
	\begin{listing}
	\begin{code}[]{java}
		@NamedQueries({
			@NamedQuery(
			name = "ReservationsViaEmail",
			query = "SELECT reservierungen FROM entity.Benutzer b WHERE b.eMail = :email"
			)
		)
	\end{code}
	
	\caption{Definition Named Query}
	\end{listing}
	\subsubsection{Aufgaben queries}
	
	\begin{enumerate}
		\item SELECT reservierungen FROM entity.Benutzer b WHERE b.eMail = :email
		\item SELECT DISTINCT b From entity.Benutzer b, entity.Zeitkarte z WHERE z.typ = entity.ZeitkartenTyp.MONATSKARTE AND b.tickets.ID = z.ID
		\item SELECT t FROM entity.Ticket t INNER JOIN t.strecke s where s not in (select strecke FROM entity.Reservierung)
	\end{enumerate}
	
	\subsubsection{Aufrufen}
		Named-queries können mittels der Session abgefragt und ausgeführt werden.
		
		
	\begin{listing}
	\begin{code}[]{java}
		List<Reservierung> reservierungs = session.getNamedQuery("ReservationsViaEmail").setParameter("email",a.geteMail()).list();
		
		Iterator<Reservierung> it = a.getReservierungen().iterator();	
	\end{code}

	\caption{Aufruf Named Query}
	\end{listing}


	